\documentclass[openany]{book}
\usepackage{lmodern}
\usepackage{amssymb,amsmath}
\usepackage{ifxetex,ifluatex}
\usepackage{fixltx2e} % provides \textsubscript
\ifnum 0\ifxetex 1\fi\ifluatex 1\fi=0 % if pdftex
  \usepackage[T1]{fontenc}
  \usepackage[utf8]{inputenc}
\else % if luatex or xelatex
  \ifxetex
    \usepackage{mathspec}
  \else
    \usepackage{fontspec}
  \fi
  \defaultfontfeatures{Ligatures=TeX,Scale=MatchLowercase}
\fi
% use upquote if available, for straight quotes in verbatim environments
\IfFileExists{upquote.sty}{\usepackage{upquote}}{}
% use microtype if available
\IfFileExists{microtype.sty}{%
\usepackage{microtype}
\UseMicrotypeSet[protrusion]{basicmath} % disable protrusion for tt fonts
}{}
\usepackage[margin=1in]{geometry}
\usepackage{hyperref}
\hypersetup{unicode=true,
            pdftitle={Singapore Society in Numbers},
            pdfauthor={Edited by Shannon Ang},
            pdfborder={0 0 0},
            breaklinks=true}
\urlstyle{same}  % don't use monospace font for urls
\usepackage{natbib}
\bibliographystyle{apalike}
\usepackage{color}
\usepackage{fancyvrb}
\newcommand{\VerbBar}{|}
\newcommand{\VERB}{\Verb[commandchars=\\\{\}]}
\DefineVerbatimEnvironment{Highlighting}{Verbatim}{commandchars=\\\{\}}
% Add ',fontsize=\small' for more characters per line
\usepackage{framed}
\definecolor{shadecolor}{RGB}{248,248,248}
\newenvironment{Shaded}{\begin{snugshade}}{\end{snugshade}}
\newcommand{\KeywordTok}[1]{\textcolor[rgb]{0.13,0.29,0.53}{\textbf{#1}}}
\newcommand{\DataTypeTok}[1]{\textcolor[rgb]{0.13,0.29,0.53}{#1}}
\newcommand{\DecValTok}[1]{\textcolor[rgb]{0.00,0.00,0.81}{#1}}
\newcommand{\BaseNTok}[1]{\textcolor[rgb]{0.00,0.00,0.81}{#1}}
\newcommand{\FloatTok}[1]{\textcolor[rgb]{0.00,0.00,0.81}{#1}}
\newcommand{\ConstantTok}[1]{\textcolor[rgb]{0.00,0.00,0.00}{#1}}
\newcommand{\CharTok}[1]{\textcolor[rgb]{0.31,0.60,0.02}{#1}}
\newcommand{\SpecialCharTok}[1]{\textcolor[rgb]{0.00,0.00,0.00}{#1}}
\newcommand{\StringTok}[1]{\textcolor[rgb]{0.31,0.60,0.02}{#1}}
\newcommand{\VerbatimStringTok}[1]{\textcolor[rgb]{0.31,0.60,0.02}{#1}}
\newcommand{\SpecialStringTok}[1]{\textcolor[rgb]{0.31,0.60,0.02}{#1}}
\newcommand{\ImportTok}[1]{#1}
\newcommand{\CommentTok}[1]{\textcolor[rgb]{0.56,0.35,0.01}{\textit{#1}}}
\newcommand{\DocumentationTok}[1]{\textcolor[rgb]{0.56,0.35,0.01}{\textbf{\textit{#1}}}}
\newcommand{\AnnotationTok}[1]{\textcolor[rgb]{0.56,0.35,0.01}{\textbf{\textit{#1}}}}
\newcommand{\CommentVarTok}[1]{\textcolor[rgb]{0.56,0.35,0.01}{\textbf{\textit{#1}}}}
\newcommand{\OtherTok}[1]{\textcolor[rgb]{0.56,0.35,0.01}{#1}}
\newcommand{\FunctionTok}[1]{\textcolor[rgb]{0.00,0.00,0.00}{#1}}
\newcommand{\VariableTok}[1]{\textcolor[rgb]{0.00,0.00,0.00}{#1}}
\newcommand{\ControlFlowTok}[1]{\textcolor[rgb]{0.13,0.29,0.53}{\textbf{#1}}}
\newcommand{\OperatorTok}[1]{\textcolor[rgb]{0.81,0.36,0.00}{\textbf{#1}}}
\newcommand{\BuiltInTok}[1]{#1}
\newcommand{\ExtensionTok}[1]{#1}
\newcommand{\PreprocessorTok}[1]{\textcolor[rgb]{0.56,0.35,0.01}{\textit{#1}}}
\newcommand{\AttributeTok}[1]{\textcolor[rgb]{0.77,0.63,0.00}{#1}}
\newcommand{\RegionMarkerTok}[1]{#1}
\newcommand{\InformationTok}[1]{\textcolor[rgb]{0.56,0.35,0.01}{\textbf{\textit{#1}}}}
\newcommand{\WarningTok}[1]{\textcolor[rgb]{0.56,0.35,0.01}{\textbf{\textit{#1}}}}
\newcommand{\AlertTok}[1]{\textcolor[rgb]{0.94,0.16,0.16}{#1}}
\newcommand{\ErrorTok}[1]{\textcolor[rgb]{0.64,0.00,0.00}{\textbf{#1}}}
\newcommand{\NormalTok}[1]{#1}
\usepackage{longtable,booktabs}
\usepackage{graphicx,grffile}
\makeatletter
\def\maxwidth{\ifdim\Gin@nat@width>\linewidth\linewidth\else\Gin@nat@width\fi}
\def\maxheight{\ifdim\Gin@nat@height>\textheight\textheight\else\Gin@nat@height\fi}
\makeatother
% Scale images if necessary, so that they will not overflow the page
% margins by default, and it is still possible to overwrite the defaults
% using explicit options in \includegraphics[width, height, ...]{}
\setkeys{Gin}{width=\maxwidth,height=\maxheight,keepaspectratio}
\IfFileExists{parskip.sty}{%
\usepackage{parskip}
}{% else
\setlength{\parindent}{0pt}
\setlength{\parskip}{6pt plus 2pt minus 1pt}
}
\setlength{\emergencystretch}{3em}  % prevent overfull lines
\providecommand{\tightlist}{%
  \setlength{\itemsep}{0pt}\setlength{\parskip}{0pt}}
\setcounter{secnumdepth}{5}
% Redefines (sub)paragraphs to behave more like sections
\ifx\paragraph\undefined\else
\let\oldparagraph\paragraph
\renewcommand{\paragraph}[1]{\oldparagraph{#1}\mbox{}}
\fi
\ifx\subparagraph\undefined\else
\let\oldsubparagraph\subparagraph
\renewcommand{\subparagraph}[1]{\oldsubparagraph{#1}\mbox{}}
\fi

%%% Use protect on footnotes to avoid problems with footnotes in titles
\let\rmarkdownfootnote\footnote%
\def\footnote{\protect\rmarkdownfootnote}

%%% Change title format to be more compact
\usepackage{titling}

% Create subtitle command for use in maketitle
\providecommand{\subtitle}[1]{
  \posttitle{
    \begin{center}\large#1\end{center}
    }
}

\setlength{\droptitle}{-2em}

  \title{Singapore Society in Numbers}
    \pretitle{\vspace{\droptitle}\centering\huge}
  \posttitle{\par}
    \author{Edited by Shannon Ang}
    \preauthor{\centering\large\emph}
  \postauthor{\par}
      \predate{\centering\large\emph}
  \postdate{\par}
    \date{Last updated 25 May 2019}

\usepackage{booktabs}
\usepackage{amsthm}
\makeatletter
\def\thm@space@setup{%
  \thm@preskip=8pt plus 2pt minus 4pt
  \thm@postskip=\thm@preskip
}
\makeatother

\begin{document}
\maketitle

{
\setcounter{tocdepth}{1}
\tableofcontents
}
\chapter*{Preface}\label{preface}
\addcontentsline{toc}{chapter}{Preface}

\begin{quote}
\textbf{Note to Readers}

This book is in Open Review. I want your feedback to make the book
better for you and other readers. To add your annotation, {select some
text} and then click the on the pop-up menu. To see the annotations of
others, click the in the upper right hand corner of the page .
\end{quote}

This online book is a compilation of resources aimed at advancing
quantitative social science in Singapore. It is meant to be a `living
document', so it will be updated as frequently as possible. The main
goal is to promote interest, rigour, and transparency in trying to
understand Singapore society through quantitative lenses. It does so by:

\begin{enumerate}
\def\labelenumi{\arabic{enumi}.}
\tightlist
\item
  \textbf{Providing information on Singapore-relevant datasets} that are
  currently used to answer research and policy questions (Chapter
  \ref{publicdata} and Chapter \ref{restricteddata}). This includes:

  \begin{itemize}
  \tightlist
  \item
    Descriptions of \emph{publicly available} datasets and how to access
    them. This overview of the `data landscape' will be helpful for
    social scientists to get started with research on Singapore, and
    prevent wasteful overlap in primary data collection across
    institutions.
  \item
    A list of \emph{restricted} or \emph{non-publicly available}
    datasets that could be used to answer important research or policy
    questions if access was granted. If available, details on the
    dataset and reasons for data restriction will also be listed. It is
    hoped that this list will promote greater transparency in data
    sharing across research teams.
  \end{itemize}
\item
  \textbf{Occasional think pieces by researchers} on best practices and
  on how to improve quantitative social science in Singapore (Chapter
  \ref{think}).
\item
  \textbf{Maintaining a repository of replicable case studies on
  Singapore society} (with annotated code, where possible) which can be
  used for illustrations in any quantitatively oriented college-level
  class (Chapter \ref{oop} onwards). These may be short summaries
  (blog-length) of published work, or side analyses that may not be
  appropriate for an academic journal but are useful for Singapore
  social science nonetheless.
\end{enumerate}

\textbf{I am actively looking for contributors} (go
\href{https://sg-numbers.netlify.com/how-to-contribute.html}{here} to
see how you can contribute). Readers with ideas on how to improve this
resource (or who may wish to help me maintain it) may use the in-built
annotation feature, or email me at
\href{mailto:shannon.ang@ntu.edu.sg}{\nolinkurl{shannon.ang@ntu.edu.sg}}.

\includegraphics{images/cc.png}\\
This book is licensed under the
\href{http://creativecommons.org/licenses/by-nc-sa/4.0/}{Creative
Commons Attribution-NonCommercial-ShareAlike 4.0 International License}.

\section*{Why I started this project}\label{why-i-started-this-project}
\addcontentsline{toc}{section}{Why I started this project}

Quantitative research is not (and should not be) the only approach we
take to understanding Singapore society, but constant appeals to ``big
data''\footnote{See, for instance,
  \url{https://www.todayonline.com/singapore/business-big-data-singapore-has-built-cutting-edge}}
or claims of ``evidence-based policy''\footnote{Government agencies such
  as the Ministry of Social and Family Development often
  \href{https://www.msf.gov.sg/about-MSF/our-people/Divisions-at-MSF/Family-Development-and-Support/Pages/default.aspx}{use
  such a phrase.}} makes it ever more important for members of the
public to \textbf{critically evaluate the use of numbers} in making
arguments or in representations of social phenomena.

Educational institutions have an important role to play in this
``data-driven'' world. Every year, undergraduates studying the social
sciences in our local universities take several courses in research
methods to fulfil the requirements of their degrees. Part of this
research methods sequence typically involves training in introductory
statistics or ``quantitative reasoning''. Quantitative courses in social
science departments differ from those taught in the natural sciences
because they are thought to be more applied - the focus is on the use of
statistical methods to answer questions about society. Understanding and
applying these methods \textbf{to the Singapore context} is crucial here
- at this point, students learn about (and hopefully are inspired by)
the kind of questions they can ask about the very society they live in,
given the quantitative tools they are learning.

However, my first exposure to statistics as an undergraduate reading
Sociology at NUS\footnote{(the) National University of Singapore} was to
textbooks containing examples from only Western societies
\citep[e.g.,][]{agresti_statistical_2009, treiman_quantitative_2009}.
While the use of these internationally-recognized textbooks may provide
some assurance of quality education, sole reliance on foreign material
often becomes a missed opportunity to inspire students to build on and
improve Singapore social science. Without contextualization\footnote{Notwithstanding
  the terribly unhelpful stereotype of social science students being
  ``good at writing but bad at numbers''.}, abstract statistical
concepts (e.g., hypotheses testing, chi-squared tests) seem removed from
everyday experience, and impede the ability to take these important
concepts beyond the classroom and into public dialogue.

I started this book with the view to use it primarily \emph{as a
teaching tool}\footnote{For instance, the public repository of
  Singapore-oriented examples and illustrations may be used to
  supplement courses based on textbooks written by international
  scholars.}, but it can be used in many other ways. In the long term, I
hope that resources in this book will encourage quantitative literacy
and research in Singapore by making it easier for interested parties to
browse, use, and understand Singapore-relevant data. Social science
researchers may use the dataset listings as a springboard for
collaboration, or contribute their own interesting case studies for the
benefit of the Singapore public. Others (such as journalists, civil
servants, or non-profit organizations) may find value in these material
as a gateway to quantitative research on Singapore society, and how to
think carefully about pertinent issues surrounding such work.

\textbf{\emph{For Singapore social science.}}

\section*{How to contribute}\label{how-to-contribute}
\addcontentsline{toc}{section}{How to contribute}

\textbf{To list a dataset}

I will progressively list any Singapore-related dataset that I know of,
but my knowledge is far from exhaustive. To help me out, you can simply
alert me to an unlisted dataset and let me know where to find
information on it (and I will write it up). However, if you could do the
following and send it to me, it would make my task much easier:

Write a short paragraph on the dataset which includes (but is not
limited to) the following information:

\begin{itemize}
\tightlist
\item
  Basic details about the dataset (name, how was data collected, how
  many observations etc.)
\item
  Name(s) of the Principal Investigator(s) (and links to their
  webpage/profiles, if possible)
\item
  How to access the dataset (e.g., a website that allows a direct
  download or lists instructions to obtain the data)
\end{itemize}

\emph{Publicly available} datasets are basically datasets that can be
downloaded freely or for which access can obtained through a simple
procedure (e.g., signing up for an online account, sending a form with a
simple research topic). \emph{Restricted} or \emph{Non-publicly
available} datasets are those that require more extensive clearance
(e.g., a background check, use of a data enclave) to access. Data for
which there is no information on access also fall under this criteria.
Email me at
\href{mailto:shannon.ang@ntu.edu.sg}{\nolinkurl{shannon.ang@ntu.edu.sg}}
to list a dataset.

\textbf{To write a think piece}

Think pieces can be of any length (suggested length is 500-2000 words)
and on any topic related to quantitative social science. For instance,
pieces may comment on the state of quantitative social science in
Singapore (e.g., what is lacking, how we can do better) and/or provide
tips for social scientists seeking to study Singapore (e.g., how to
write grant applications, where to find data). That said, I am still
mulling over whether this should be an invite-only section, or have
pieces go through some kind of review mechanism (I do not wish to be the
sole arbiter of what goes up here). Nevertheless, email me at
\href{mailto:shannon.ang@ntu.edu.sg}{\nolinkurl{shannon.ang@ntu.edu.sg}}
if you think you can contribute a think piece (or have someone to
suggest).

\textbf{To contribute a case study}

Case studies are meant to illustrate a point about Singapore as a
society or quantitative methods in general. These may include
blog-length summaries of published research, or smaller side analyses
that are useful for Singapore social science. I am hoping that most of
these case studies include some form of data analysis, and/or relate a
useful (quantitative) concept to the Singapore context. Replicable case
studies are preferred (i.e., anyone familiar with some code\footnote{Code
  can be in any language - R, Stata, Python, SAS, Mplus etc.} can
reproduce results), but researchers unable to share code (and data)
should not disqualify themselves from contributing. Further, this online
book is written using the Rmarkdown language, so it would be helpful if
contributors are familiar with it - but this is not a prerequisite.
Email me at
\href{mailto:shannon.ang@ntu.edu.sg}{\nolinkurl{shannon.ang@ntu.edu.sg}}
if you have an idea for a case study, and we can work together to make
it happen.

\section*{Acknowledgements}\label{acknowledgements}
\addcontentsline{toc}{section}{Acknowledgements}

This book is being written through the \textbf{bookdown} package
\citep{R-bookdown}, which was built on top of R Markdown and
\textbf{knitr} \citep{xie2015}.

Contributors other than me include:

\section*{About me}\label{about-me}
\addcontentsline{toc}{section}{About me}

Little write-up about myself, which I will insert in due course\ldots{}
For now, refer to \href{https://www.shannonang.net}{my personal
website}.

\part{Datasets for Social
Science}\label{part-datasets-for-social-science}

\chapter{Public Data}\label{publicdata}

This section contains a list of public datasets available for social
scientists to analyze. Datasets should be in disaggregated form in a way
that is useful for academic work\footnote{In essence, academics should
  be able to publish from it.} and social research. For each one, a
brief description of the dataset, the investigators, and details for
access will be included. It is expected that this list will continue to
grow.

Aggregated administrative data, like those that can be found on
\url{https://data.gov.sg} or from government reports, are not really the
focus here. \emph{However}, information about data that can be linked to
disaggregated datasets (e.g., data on neighbourhood characteristics, or
other forms of contextual data) to improve analyses are welcomed.

Readers who believe I have provided wrong information on any of these
datasets can \href{shannon.ang@ntu.edu.sg}{contact me}, so that I can
correct it. Use the in-built annotation feature, or send me an email at
\href{mailto:shannon.ang@ntu.edu.sg}{\nolinkurl{shannon.ang@ntu.edu.sg}}
if you know about a dataset that should be featured in this list but is
not included here.

\section{Asian Barometer Survey (ABS)}\label{abs}

The Asian Barometer Survey is a cross-country, longitudinal study
(repeated cross-section) of public opinion in about 21 countries. Data
was collected through face-to-face interviews with adults aged 21 and
above. The survey assesses opinions on issues such as political values,
democracy, governance, human security, and economic reform. Singapore
has participated in 3 out of 4 waves of data collection - 2006 (N=1012),
2010 (N=1000), and 2014 (N=1039).

ABS is coordinated by the
\href{http://www.asianbarometer.org/structure}{Hu Fu Center for East
Asia Democratic Studies} at the National Taiwan University.

Instructions for data access can be found at
\url{http://www.asianbarometer.org/data/data-release}.

\section{Panel on Health and Ageing of Singaporean Elderly
(PHASE)}\label{phase}

The Panel on Health and Ageing of Singaporean Elderly is a longitudinal
panel study that tracks changes in the physical, social and mental
health of Singapore residents. Respondents are aged 60 years and above
in 2009, and three waves of data collection have been conducted so far.
Wave 1 was conducted in 2009 (N=4990), Wave 2 in 2011 (N=3103), and Wave
3 in 2015 (N=1572).

PHASE includes data on physical health, mental health, social engagement
(e.g., loneliness, social participation, social networks), income,
employment, and housing. Anthropometric and performance measurements,
including blood pressure, sitting and standing height, waist
circumference, body weight and hand grip strength, were also conducted.

Principal Investigators for PHASE are Associate Professor
\href{https://www.duke-nus.edu.sg/hssr/our-team/faculty/faculty-staff-details/Detail/13200}{Angelique
Chan}, Professor
\href{https://www.duke-nus.edu.sg/hssr/our-team/faculty/faculty-staff-details/Detail/31920}{David
Matchar}, and Assistant Professor
\href{https://www.duke-nus.edu.sg/hssr/our-team/faculty/faculty-staff-details/Detail/30743}{Rahul
Malhotra} at Duke-NUS Medical School.

Instructions for data access can be found at
\url{https://www.duke-nus.edu.sg/care/research/dataset-codebook}.

\section{World Values Survey (WVS)}\label{wvs}

The World Values Survey is a global study to help social scientists
understand changes in the beliefs, values and motivations of people
across multiple countries. To date, there have been 7 waves of data
collection (repeated cross-section, not panel) from almost 100
countries. The survey includes questions on societal trust, religion,
work, security, and politics. Singapore participated in Wave 4 (2002,
N=1512) and Wave 6 (2012, N=1972). Principal Investigators are Associate
Professor \href{http://profile.nus.edu.sg/fass/soctanes/}{Tan Ern Ser}
at NUS (2002), and Associate Professor
\href{https://www.suss.edu.sg/about-suss/faculty-and-staff/detail/vincent-chua}{Vincent
CH Chua} at SUSS (2012).

Data can be accessed at \url{http://www.worldvaluessurvey.org}.

\chapter{Restricted Data}\label{restricteddata}

This section contains a list of datasets that are potentially useful for
social scientists to analyze, but for which access is restricted. Other
than a description of the data (as far as possible), this will also
include information on how restrictions may be lifted (i.e., how to gain
access). It is hoped that listing them here will promote transparency in
data sharing across research teams, and eventually prevent wasteful
overlap in primary data collection across institutions. As in the public
data section, disaggregated data that can be used for research is the
focus here.

Readers who believe I have wrongly listed a dataset here (i.e., they are
not restricted), or have more accurate information than provided, can
\href{shannon.ang@ntu.edu.sg}{contact me}. Use the in-built annotation
feature, or send me an email at
\href{mailto:shannon.ang@ntu.edu.sg}{\nolinkurl{shannon.ang@ntu.edu.sg}}
if you know about a dataset that should be featured in this list but is
not included here.

\section{Retirement and Health Study (RHS)}\label{rhs}

The RHS is a longitudinal survey of Singapore residents' retirement and
healthcare needs and how they change over time. It is conducted by the
Central Provident Fund Board (CPF). This is a panel study of individuals
aged 45 to 85 in 2014, with the same individuals being interviewed once
every two years (for ten years, beginning in 2014). The survey includes
information on household expenses, employment, health, and financial
status. This is a large study with potentially many uses - RHS purports
to have reached out to more than 23,000 participants in the first two
rounds of interviews\footnote{\url{https://www.cpf.gov.sg/Assets/members/Documents/RHS_FAQ_Booklet.pdf}}.

More information is available at
\url{https://www.cpf.gov.sg/Members/Others/member-pages/retirement-and-health-study-(rhs)}.
The RHS website does not list any plans to make the dataset publicly
available, and is accessible only to government agencies or researchers
working with/for government agencies. The RHS study team can be reached
at \href{mailto:cpf_rhs@cpf.gov.sg}{\nolinkurl{cpf\_rhs@cpf.gov.sg}}.

\section{Singapore Life Panel (SLP)}\label{slp}

The Singapore Life Panel administers an internet-based monthly survey to
approximately 11,000 Singaporeans aged 50 to 70 years. Information on
income, expenditure, health, work and housing choices are solicited from
panel members. The panel survey is one of the largest
population-representative monthly surveys conducted in the world. This
study is unique in Singapore because of the monthly frequency of surveys
- the sheer amount of data collected over time is unprecedented.

The Singapore Life Panel is housed at the
\href{https://crea.smu.edu.sg/singapore-monthly-panel}{Centre for
Research on the Economics of Ageing} at SMU. The SLP website does not
list any plans to make the dataset publicly available outside of their
own research team. The SLP team can be reached at
\href{mailto:crea@smu.edu.sg}{\nolinkurl{crea@smu.edu.sg}}.

\section{Singapore Panel Study on Social Dynamics (SPSSD)}\label{spssd}

The Singapore Panel Study on Social Dynamics was started in 2014 to
track changes in the people's lives over time. It is a longitudinal
panel study, starting with 5002 heads of households interviewed in 2014.
Its purpose is to measure family dynamics, societal values and attitudes
relevant to national identity and social mobility over time. Waves 1 to
4 of data collection have been completed, with Wave 5 beginning in 2019.
A strength of this study is its panel data - few Singapore datasets with
rich data on family dynamics have this many waves collected from the
same individuals over time.

The Principal Investigator for this study is Dr
\href{http://lkyspp.nus.edu.sg/our-people/faculty/leong-chan-hoong}{Leong
Chan Hoong} at the Institute of Policy Studies, NUS. The SPSSD website
does not list any plans to make the dataset publicly available, probably
because of restrictions imposed by government funders. The SPSSD study
team can be reached at
\href{mailto:ips.soclab@nus.edu.sg}{\nolinkurl{ips.soclab@nus.edu.sg}}.

\part{Think Pieces}\label{part-think-pieces}

\chapter{Thinking about Numbers}\label{think}

Think pieces section. Empty for now, but would you like to contribute?
Email me at
\href{mailto:shannon.ang@ntu.edu.sg}{\nolinkurl{shannon.ang@ntu.edu.sg}}.

\part{Case Studies}\label{part-case-studies}

\chapter{Blown out of proportion}\label{oop}

\begin{verbatim}
Contributor: Shannon Ang
Date: 21 May 2019
\end{verbatim}

Proportions (sometimes expressed in percentages) are commonly used in
popular media to reflect public opinion. In fact, it is often the only
type of statistic we get\footnote{Reports such as those released in the
  form of IPS working papers, sometimes include multivariable analysis,
  but often after many pages of crosstabulations} to evaluate as
``evidence''. For instance, a news article may state that ``nearly 46
per cent of those aged 18 to 25 would allow extremist views that deem
all other religions as enemies to be published''\footnote{\url{https://www.todayonline.com/singapore/nearly-1-2-young-sporeans-open-extremist-views-being-posted-online-survey-shows}},
or that ``59 per cent of Chinese find a Malay president
acceptable''\footnote{\url{https://www.straitstimes.com/singapore/majority-willing-to-accept-president-or-pm-of-another-race-but-prefer-one-of-their-own}}.
While these proportions are easy for the general public to understand,
they can be misleading if not read carefully. This case study looks at
two different news articles, showing how some claims can be exaggerated
by careless use of numbers.

\section{Media claim 1: Support for the Watain ban}\label{watain}

Swedish black metal band Watain was supposed to perform in Singapore on
7 March 2019. However, the gig was cancelled just hours before it was
scheduled to begin, with the government citing concerns from the
Christian community\footnote{See
  \url{https://www.channelnewsasia.com/news/singapore/watain-concert-cancelled-christian-community-reaction-shanmugam-11399434}}.
To evaluate public sentiment towards this incident, REACH\footnote{The
  Singapore Government's feedback unit} conducted a poll with 680
Singaporeans aged 15 and above. Of interest here is how results from
this poll was represented in public discourse.

\begin{quote}
Our assessment of public sentiment turned out to be correct, because a
subsequent REACH survey showed that, first of all, that 60\% were aware
of the cancellation. \textbf{Of those who were aware}, 86\% of
Christians agreed with the cancellation. That I think will be natural.
But 64\% \textbf{of all who had heard about the cancellation}, Christian
and non-Christian, also agreed with the cancellation. Twenty-eight
percent thought that it should not have been cancelled.

Minister for Home Affairs K Shanmugam, 1 April 2019, emphasis mine
\end{quote}

The quote above is
\href{https://sprs.parl.gov.sg/search/sprs3topic?reportid=ministerial-statement-1170}{taken
directly from the Hansard}, and is consistent with the results shown in
REACH's
\href{https://www.reach.gov.sg/~/media/2019/press-release/findings-of-poll-on-watain-concert--1-april-2019.pdf}{press
release}. Note the phrases that I bolded for our purposes, which I will
call ``qualifiers''.

\begin{figure}

{\centering \includegraphics[width=0.8\linewidth]{images/proportion/STwatain} 

}

\caption{Screenshot of online article on results from REACH poll. Retrieved May 21, 2019.}\label{fig:st-watain}
\end{figure}

The next day, national newspaper The Straits Times ran
\href{https://www.straitstimes.com/politics/singapolitics/parliament-two-out-of-three-singaporeans-back-governments-move-to-cancel}{a
story} headlined ``Parliament: Two in three back move to ban Watain
gig''. Within the text of the article, it reads:

\begin{quote}
The Government decided to cancel the permit for Watain's concert last
month when it received reports that mainstream Christians were very
concerned and offended by the band, Home Affairs Minister K. Shanmugam
said yesterday. \textbf{And a survey of Singaporeans by government
feedback unit Reach found that two in three supported the move, he
noted.} Among Christians, 86 per cent were supportive of the move to
disallow the concert, the Reach poll found.
\end{quote}

Note the qualifier ``among those who were aware'' is neither in the
headline (Figure \ref{fig:st-watain}) nor the body of the
article\footnote{CNA ran a similar headline, but included the qualifier
  within the article. See
  \url{https://www.channelnewsasia.com/news/singapore/2-in-3-singaporeans-in-reach-poll-supported-government-s-11401066}}.
\emph{Why is this important?}
\href{https://www.reach.gov.sg/~/media/2019/press-release/findings-of-poll-on-watain-concert--1-april-2019.pdf}{Results
from REACH} show that 63\% of respondents were aware, and \emph{out of
these respondents}, 64\% supported the government's ban. This means that
out of \emph{all} respondents to the survey, only about 40\% reported
supporting the ban. The qualifying phrase ``among those who were aware''
meaningfully changes the interpretation of the results - we shouldn't be
able to say that \textbf{the majority of Singaporeans} supported the ban
when in fact only 40\% of the survey respondents did so.

In effect, the Straits Times article is invoking a strong assumption
here (see \ref{ooptech} for a more technical explanation) - that
\emph{if} those who were unaware were in fact able to express their
support for the ban, the same proportion of respondents (among those who
were aware, 64\%) would also support the ban. But being aware of the ban
is a \emph{prerequisite} for support of the ban, which makes this
assumption rather unreasonable. Even assuming this hypothetical scenario
were possible, the actual figure could be higher or lower - it depends
on how similar (or different) the unaware are to the aware. Those who
were not aware may be less likely to care about black metal music (or
simply too busy to keep up with current affairs) and simply base their
support of the ban on their general sentiment toward government
policies. This seemingly minor omission of the qualifier can lead to
false conclusions pretty quickly. Let us look at another example.

\section{Media claim 2: Web-savvy Seniors}\label{websavvy}

Part of my research involves looking at how Internet use can improve the
lives of older adults \citep[see][]{ang_going_2018}. I was interested in
what the overall situation was like in Singapore, and googled something
like ``internet use seniors''. One 2014 article in the Straits Times
immediately caught my eye (see Figure \ref{fig:st-websavvyseniors}).

\begin{figure}

{\centering \includegraphics[width=0.8\linewidth]{images/proportion/STwebsavvyseniors} 

}

\caption{Screenshot of online article on web-savvy seniors. Retrieved May 21, 2019.}\label{fig:st-websavvyseniors}
\end{figure}

Within the article, the reporter states:

\begin{quote}
Also, \textbf{78 per cent of those aged 55 and older here access the
Internet every day} either via the traditional Web browser or smartphone
apps, putting Singapore fifth in the world for having the most
Internet-savvy seniors.
\end{quote}

I was skeptical. Over and above my anecdotal experience with Singapore
older adults, past research in the United States\footnote{For instance,
  see
  \url{https://www.pewinternet.org/2012/06/06/older-adults-and-internet-use/}}
gave me reason to expect that the proportion of older people even using
the Internet (everyday or not) should be much lower. Some results from
Consumer Barometer are available online, so we can
\href{https://www.consumerbarometer.com/en/trending/?countryCode=SG\&category=TRN-AGE-55-PLUS}{check
for ourselves}. Of interest in Figure \ref{fig:cb-seniorsinternet} are
numbers reflecting internet use in year 2014, which is when the news
article was published. Note that the percent of Singaporeans aged 55 and
above who use the internet daily in 2014 \textbf{is 29\%, not 78\% as
the article suggests.}

\begin{figure}

{\centering \includegraphics[width=1\linewidth]{images/proportion/CBseniorsinternet} 

}

\caption{Screenshot of Consumer Barometer findings across time. Retrieved May 21, 2019.}\label{fig:cb-seniorsinternet}
\end{figure}

How then, did the reporter get things so wrong? While detailed
statistics for 2014 doesn't seem available online anymore, a little
investigation using 2017 figures shows how the reporter arrived at a
number as high as 78\%.

\begin{figure}

{\centering \includegraphics[width=0.9\linewidth]{images/proportion/CBsinglequestion} 

}

\caption{Screenshot of Consumer Barometer results on 2017 internet use. Retrieved May 21, 2019.}\label{fig:cb-single}
\end{figure}

Looking at Figure \ref{fig:cb-single}, the crucial part is the footnote
that says ``base'', which tells us that the 82\% figure for daily
Internet usage in 2017 are \textbf{among those who use the internet}. We
can easily calculate this 82\% with the numbers in Figure
\ref{fig:cb-seniorsinternet} - note that 43\footnote{The percent of
  Singapore residents aged 55+ using the internet in 2017, from Figure
  \ref{fig:cb-seniorsinternet}} is approximately 82\% of 53\footnote{The
  percent of Singapore residents aged 55+ using the internet
  \emph{daily} in 2017, from Figure \ref{fig:cb-seniorsinternet}}. That
is, \(\frac{43}{53} \approx 0.82\). This recovers the 82\% figure that
we see in Figure \ref{fig:cb-single}.

Using the same strategy, we can recover the reporter's figure for 2014:
\(\frac{29}{37} \approx 0.78\).

\emph{What does this mean?} This means that just like the reporter in
the Watain example (\ref{watain}), this reporter left out an important
qualifier - only 29\% of all older adults in Singapore use the internet
daily, but 78\% \textbf{of those who use the internet} use it daily.
This vast discrepancy is highly consequential - the statement that ``78
per cent of those aged 55 and older here access the Internet every day''
is false, and the headline that ``Majority of Singapore seniors are Web
savvy'' is misleading at best.

\section{Technical Appendix}\label{ooptech}

Before we go into a more technical explanation of what went wrong in
these two cases, let us first move from proportions to probabilities.
The difference between a proportion and a probability is important here.
Note that when Minister Shanmugam asserted the REACH poll provided
evidence that the Government's ``assessment of public sentiment turned
out to be correct'', he was not suggesting that 680 Singaporeans form
the whole Singapore public. The underlying assumption was that since
most survey respondents (who were aware) supported the ban, it is likely
that most Singaporeans (who are aware) will also support the ban. That
is, he was using the \emph{proportion} of supportive survey respondents
(a description of the sample), to infer the \emph{probability} (a
hypothetical quantity) of any one Singaporean supporting the ban.

The difference between a probability and a proportion may be simplified
using a coin flip example. If I flip a fair coin 4 times, the proportion
of heads may be 0, 0.25, 0.5, 0.75, or 1. However, since it is a fair
coin, the probability of getting a heads is, by definition, 0.5. So the
proportion may or may not equal the probability. What we know is that
the more times I flip the coin, the more likely the proportion of heads
will reflect the true probability of getting a heads. It is thus common
to hear people say that the probability is the ``long-run proportion of
an event''. Below is some code (in R) for you to try out the coin flip
example.

\begin{Shaded}
\begin{Highlighting}[]
\CommentTok{# Set the number of trials to 4. }
\CommentTok{# You may change the number to see what happens.}
\NormalTok{n <-}\StringTok{ }\DecValTok{4} 
\CommentTok{# Get the proportion of heads after flipping a fair coin n times. }
\CommentTok{# Try this a few times.}
\KeywordTok{sum}\NormalTok{(}\KeywordTok{rbinom}\NormalTok{(n, }\DecValTok{1}\NormalTok{, }\DataTypeTok{prob=}\FloatTok{0.5}\NormalTok{))}\OperatorTok{/}\NormalTok{n }
\end{Highlighting}
\end{Shaded}

We have now established that the main reason why we are interested in
proportions from a REACH poll is because they purport to tell us
something about Singaporeans in general. That is, the REACH poll
suggests that if we were to randomly pick a Singaporean from those who
are aware of the ban, the probability of this person supporting the ban
is about 0.64 (or 64\%). The problem at hand then reduces to a trivial
probability question, assuming that we all remember basic probability
rules from secondary (primary?) school\footnote{Or that we can Google it
  if not}. If the REACH poll is indeed representative of all
Singaporeans, then we have the following quantities:

 \[
\begin{aligned}
&\Pr(\text{Aware of Ban}) = 0.63 \\
&\Pr(\text{Not Aware of Ban}) = 1 - \Pr(\text{Aware of Ban}) = 0.37 \\
&\Pr(\text{Support Ban } | \text{ Aware of Ban}) = 0.64 
\end{aligned}
\]

\(\Pr(\text{Support Ban } | \text{ Aware of Ban})\) is a conditional
probability, but the quantity that is being asserted in the news article
is \(\Pr(\text{Support Ban})\), which is the total probability. Using
the law of total probability, we know that:

 \[
\begin{aligned}
\Pr(\text{Support Ban}) &= \Pr(\text{Support Ban } | \text{ Aware of Ban})\cdot \Pr(\text{Aware of Ban}) \\
& \quad + \Pr(\text{Support Ban } | \text{ Not Aware of Ban})\cdot \Pr(\text{Not Aware of Ban}) 
\end{aligned}
\]

Plugging in the numbers that we have,

 \[
\begin{aligned}
\Pr(\text{Support Ban}) &= 0.64 \cdot 0.63 + \Pr(\text{Support Ban } | \text{ Not Aware of Ban}) \cdot 0.37 \\
\end{aligned}
\]

we see that \(\Pr(\text{Support Ban}) = 0.64\) if and only if
\(\Pr(\text{Support Ban } | \text{ Not Aware of Ban})\) also equals
\(0.64\). That said,
\(\Pr(\text{Support Ban } | \text{ Not Aware of Ban})\) is logically
impossible, and should equal zero. Similarly, for the Web-savvy Seniors
example,

 \[
\begin{aligned}
\Pr(\text{Use Internet Daily}) &= \Pr(\text{Use Internet Daily } | \text{ Use Internet})\cdot \Pr(\text{Use Internet}) \\
& \quad + \Pr(\text{Use Internet Daily } | \text{ Don't Use Internet})\cdot \Pr(\text{Don't Use Internet}) \\
&= 0.78 \cdot 0.37 + \Pr(\text{Use Internet Daily } | \text{ Don't Use Internet})\cdot 0.63 \\
\end{aligned}
\]

where \(\Pr(\text{Use Internet Daily } | \text{ Don't Use Internet})\)
is impossible and should be zero. In both cases, total probabilites are
substantially different from the conditional probabilities, and there is
no reason to believe they would be the same.

\section{Conclusion}\label{conclusion}

By now, it should be clear that qualifiers attached to proportions (and
percentages) are critical. Without them, results from studies can be
blown out of proportion. It is not wise to completely rely on assertions
made by news articles (or other kinds of reports), even from supposedly
credible agencies like the Straits Times. As we have seen, social
scientists should be comfortable with interpreting data from its
source\footnote{This, however, first requires data to be made available
  for replication purposes.} in order to evaluate claims that are being
made in public discourse today.

\chapter{Are we lonely?}\label{lonely}

\begin{verbatim}
Contributor: Shannon Ang
Date: 25 May 2019
\end{verbatim}

In population health research, there are a number of commonly used
scales to measure psychosocial well-being. For instance, the
\href{https://www.apa.org/pi/about/publications/caregivers/practice-settings/assessment/tools/depression-scale}{Center
for Epidemiologic Studies Depression Scale} (CES-D) is widely used to
measure depression, and the
\href{https://euroqol.org/eq-5d-instruments/}{EQ-5D}\footnote{There's no
  `full' name for this, its just referred to as the EQ-5D.} to measure
quality of life. These scales seldom have a intuitive interpretation -
who knows what 10 points on the CES-D scale actually means in `real
life', versus 12 points? To address this, social scientists often choose
a ``cut-off'' point to simplify the measure into two categories (e.g.,
either you are depressed, or you are not). Some of these cut-off points
are well researched (such the cut-off point for mild cognitive
impairment), while others are more arbitrary.

This case study looks at the prevalence of loneliness in Singapore older
adults, and how these cut-offs can shape the way we think about it. The
focus here is \emph{not} to criticize researchers' choices of cut-off
points. Instead, this case study seeks to provide a way to evaluate
claims that based on these cut-offs, so that we understand how to
compare claims across studies and/or reports.

\section{The lonely dichotomy}\label{the-lonely-dichotomy}

Loneliness is a real issue for many people today, and it has been shown
to have deleterious effects on health
\citep{rico-uribe_association_2018}. More of us are beginning to realize
that social connections are key, especially for older persons. The
Straits Times carried an
\href{https://www.straitstimes.com/singapore/living-with-family-but-still-feeling-lonely}{article}
in 2018, with the headline ``Senior citizens living with family, but
still feeling lonely''.

\begin{figure}

{\centering \includegraphics[width=0.8\linewidth]{images/loneliness/livingarr} 

}

\caption{Screenshot of online article on lonely seniors. Retrieved May 25, 2019.}\label{fig:st-lonely}
\end{figure}

But what does \textbf{feeling lonely} really mean? There seems to be a
dichotomy being drawn here - either you feel lonely, or you don't feel
lonely. There is little room for nuance like ``I feel lonely when I ride
the bus by myself''. Of course, some level of simplification is needed
to compare across groups - and this simplification is what we need to
examine.

The news article highlights studies on loneliness done by two different
groups - a team at the National Healthcare Group (NHG), and a team at
the Centre for Ageing Research and Education (CARE) in Duke-NUS Medical
School\footnote{Led by Associate Professor
  \href{https://www.duke-nus.edu.sg/hssr/our-team/faculty/faculty-staff-details/Detail/13200}{Angelique
  Chan}}. As we will see, this lonely/not lonely dichotomy is drawn by
reseachers as well, but sometimes \textbf{in completely different ways}.
Let us take a closer look.

\section{Lonely by whose standard}\label{lonely-by-whose-standard}

The studies of interest\footnote{All these studies are open-access
  articles, anyone can access them even without a library subscription.}
here are \citet{wee_loneliness_2019}, \citet{ge_social_2017},
\citet{lim_association_2017}, and \citet{chan_loneliness_2015}. All of
these studies use a variant of the 3-item UCLA Loneliness Scale
\citep{hughes_short_2004}. The 3-item UCLA loneliness scale consists of
three questions, which all of these studies use:

\begin{enumerate}
\def\labelenumi{\arabic{enumi}.}
\tightlist
\item
  How often do you feel you lack companionship?
\item
  How often do you feel left out?
\item
  How often do you feel isolated from others?
\end{enumerate}

For each of the questions listed above, respondents were given a range
of responses to choose from. These are listed in Figure
\ref{fig:st-lonely-graph}. As you might notice, they were different
across the studies. \citet{wee_loneliness_2019} and
\citet{ge_social_2017} had only 3 response categories (the right
column), while \citet{lim_association_2017} and
\citet{chan_loneliness_2015} had 5 different response categories (the
left column). I put the point values for each response in parentheses.
In all of these studies, researchers added up these points across the 3
questions, and came up with their own `cut-off' point to determine who
was ``lonely''.

\begin{figure}

{\centering \includegraphics[width=0.5\linewidth]{images/loneliness/lonely_graph} 

}

\caption{Summary of response categories}\label{fig:st-lonely-graph}
\end{figure}

I added the \textbf{blue arrows} in Figure \ref{fig:st-lonely-graph} to
show how the 5 category option can be mapped to the 3 category option
(but not vice versa). We don't really know what kind of bias this will
introduce\footnote{For instance, one might argue that respondents might
  choose differently when given more gradational categories.}, but at
face value I think this looks pretty reasonable. Why am I doing this?
This ``matching'' allows us to do a little experiment with publicly
available data\footnote{So you can try it yourself} to answer the
following question: \textbf{How does changing the criteria change our
view of loneliness in Singapore?} The next section compares these
different coding schemes.

\section{Same data, different
results}\label{same-data-different-results}

For this simple analysis, I use data from Wave 2 of the Panel on Health
and Ageing of Singaporean Elderly (PHASE) (see \ref{phase}), conducted
in 2011. This is a nationally representative study of older adults aged
60 and above. It is essentially the same dataset used in
\citet{lim_association_2017} and \citet{chan_loneliness_2015}. Code
provided is in R. For the sake of brevity, I leave out observations with
any missing values on any of the loneliness items. A key concern of the
news article in Figure \ref{fig:st-lonely} is that even older adults
living with family members may be lonely, so we will look at a
cross-tabulation of living arrangements with ``loneliness''.

\textbf{Coding scheme 1: Lots of loneliness}

I first follow the coding scheme in \citet{lim_association_2017} and
\citet{chan_loneliness_2015}. I sum the items up (giving me a score that
ranges from 0-12), and then dichotomize respondents into people who are
``not lonely'' (score of 0), and those who are ``lonely'' (score of
1-12)\footnote{Note that \citet{chan_loneliness_2015} further splits the
  ``lonely'' group into ``sometimes lonely'' and ``mostly lonely''.
  \citet{lim_association_2017}, however, does not make this distinction.
  I have grouped them together since this is the way that it has usually
  been represented in public discourse (e.g., the claim that those who
  are lonely have a higher risk of mortality compared to those not
  lonely. See, for instance,
  \url{https://www.straitstimes.com/singapore/those-who-feel-lonely-more-prone})}.
This cut-off point seems to have arisen from a ``common-sense'' approach
rather than any kind of formal testing - that is, group the people who
never experience loneliness in one group, and then put the rest who have
had some experience of loneliness in another.

\begin{Shaded}
\begin{Highlighting}[]
\CommentTok{# Load required libraries}
\KeywordTok{library}\NormalTok{(dplyr)}

\CommentTok{# Note: You first need to read in the data}
\CommentTok{# The data already contains a pre-coded version according to these criteria}
\NormalTok{lonely1_cat <-}\StringTok{ }\NormalTok{phase}\OperatorTok{$}\NormalTok{w2_loneliness_yesno }\OperatorTok
\StringTok{    }\KeywordTok{factor}\NormalTok{(}\DataTypeTok{labels=}\KeywordTok{c}\NormalTok{(}\StringTok{"Not lonely"}\NormalTok{, }\StringTok{"Lonely"}\NormalTok{))}
\CommentTok{# Make a table (proportions are weighted to account for survey design)}
\NormalTok{knitr}\OperatorTok{::}\KeywordTok{kable}\NormalTok{(}
\NormalTok{  GDAtools}\OperatorTok{::}\KeywordTok{prop.wtable}\NormalTok{(livingarr, lonely1_cat, }
                      \DataTypeTok{dir=}\DecValTok{1}\NormalTok{, }\DataTypeTok{digits=}\DecValTok{3}\NormalTok{, }\DataTypeTok{w=}\NormalTok{phase}\OperatorTok{$}\NormalTok{w2_weights, }\DataTypeTok{na=}\NormalTok{F, }\DataTypeTok{mar=}\NormalTok{F), }
  \DataTypeTok{caption =} \KeywordTok{paste0}\NormalTok{(}\StringTok{'Crosstabulation using criteria in Lim and Chan (2017)'}\NormalTok{,}
                   \StringTok{'and Chan et al (2015). Note that these are row percentages.'}\NormalTok{),}
  \DataTypeTok{booktabs =} \OtherTok{TRUE}\NormalTok{)}
\end{Highlighting}
\end{Shaded}

\begin{table}[t]

\caption{\label{tab:table-lonely1}Crosstabulation using criteria in Lim and Chan (2017)and Chan et al (2015). Note that these are row percentages.}
\centering
\begin{tabular}{lrr}
\toprule
  & Not lonely & Lonely\\
\midrule
Living alone & 40.811 & 59.189\\
Living with spouse only & 67.844 & 32.156\\
Living with child only & 49.871 & 50.129\\
Living with spouse and child & 71.712 & 28.288\\
Living with others only & 44.918 & 55.082\\
\bottomrule
\end{tabular}
\end{table}

Table \ref{tab:table-lonely1} gives me a similar proportion as suggested
in the news article - that is,

\begin{quote}
``{[}Associate Professor Chan's study{]} found that half of Singaporeans
over 60 felt lonely some or most of the time. But those who lived with
spouses, or with spouses and children, did not.''
\end{quote}

 These numbers are indeed worrying. Older adults living alone are
understandly lonely, but those who live with their children (but without
their spouse) are not that far behind (50.1\%!). Even 28\% of those who
live with their spouse and child feel lonely, like the news article
(Figure \ref{fig:st-lonely}) suggests.

\textbf{Coding scheme 2: Not that much loneliness}

We then arrive at the coding scheme used by \citet{wee_loneliness_2019}
and \citet{ge_social_2017}. Summing the items gives me a score that
ranges from 3-9, and I then dichotomize the group into people who are
``not lonely'' (score of 3-5), and those who are ``lonely'' (score of
6-9). Note that these cut-points are probably arbitrary - while the
researchers cite a paper each to justify their use of the cut-point, the
cited papers do not really provide evidence in support of the cut-point.
The closest support for the cut-point in the cited papers that I could
discern is in \citet{steptoe_social_2013}, which states that they used
the top quintile\footnote{In their sample, not the Singapore one.} to
define loneliness. No reason was given as to why the top quintile was
chosen. Table \ref{tab:table-lonely2} shows the distribution of
``loneliness'' according to these criteria.

\begin{Shaded}
\begin{Highlighting}[]
\CommentTok{# Recode and sum the loneliness scores}
\NormalTok{lonely2 <-}\StringTok{ }\NormalTok{phase }\OperatorTok\StringTok{ }
\StringTok{  }\KeywordTok{select}\NormalTok{(w2_Q10_1_GV1, w2_Q10_2_GV1, w2_Q10_3_GV1) }\OperatorTok\StringTok{  }
\StringTok{  }\KeywordTok{mutate_all}\NormalTok{(}\KeywordTok{funs}\NormalTok{(}\KeywordTok{recode}\NormalTok{(.,}\StringTok{`}\DataTypeTok{0}\StringTok{`}\NormalTok{ =}\StringTok{ }\DecValTok{1}\NormalTok{, }\StringTok{`}\DataTypeTok{1}\StringTok{`}\NormalTok{ =}\StringTok{ }\DecValTok{1}\NormalTok{, }\StringTok{`}\DataTypeTok{2}\StringTok{`}\NormalTok{ =}\StringTok{ }\DecValTok{2}\NormalTok{, }\StringTok{`}\DataTypeTok{3}\StringTok{`}\NormalTok{ =}\StringTok{ }\DecValTok{3}\NormalTok{, }\StringTok{`}\DataTypeTok{4}\StringTok{`}\NormalTok{ =}\StringTok{ }\DecValTok{3}\NormalTok{))) }\OperatorTok
\StringTok{  }\KeywordTok{rowSums}\NormalTok{()}

\CommentTok{# Categorize according to cut-off point}
\NormalTok{lonely2_cat <-}\StringTok{ }\KeywordTok{if_else}\NormalTok{(lonely2 }\OperatorTok{<}\StringTok{ }\DecValTok{6}\NormalTok{, }\DecValTok{0}\NormalTok{, }\DecValTok{1}\NormalTok{) }\OperatorTok\StringTok{ }
\StringTok{  }\KeywordTok{factor}\NormalTok{(}\DataTypeTok{labels=}\KeywordTok{c}\NormalTok{(}\StringTok{"Not lonely"}\NormalTok{, }\StringTok{"Lonely"}\NormalTok{))}

\CommentTok{# Show table}
\NormalTok{knitr}\OperatorTok{::}\KeywordTok{kable}\NormalTok{(}
\NormalTok{  GDAtools}\OperatorTok{::}\KeywordTok{prop.wtable}\NormalTok{(livingarr, lonely2_cat, }
                      \DataTypeTok{dir=}\DecValTok{1}\NormalTok{, }\DataTypeTok{digits=}\DecValTok{3}\NormalTok{, }\DataTypeTok{w=}\NormalTok{phase}\OperatorTok{$}\NormalTok{w2_weights, }\DataTypeTok{na=}\NormalTok{F, }\DataTypeTok{mar=}\NormalTok{F), }
  \DataTypeTok{caption =} \KeywordTok{paste0}\NormalTok{(}\StringTok{'Crosstabulation using criteria in Wee et al (2019)'}\NormalTok{,}
            \StringTok{'and Ge et al (2017). Note that these are row percentages.'}\NormalTok{),}
  \DataTypeTok{booktabs =} \OtherTok{TRUE}\NormalTok{)}
\end{Highlighting}
\end{Shaded}

\begin{table}[t]

\caption{\label{tab:table-lonely2}Crosstabulation using criteria in Wee et al (2019)and Ge et al (2017). Note that these are row percentages.}
\centering
\begin{tabular}{lrr}
\toprule
  & Not lonely & Lonely\\
\midrule
Living alone & 84.419 & 15.581\\
Living with spouse only & 96.555 & 3.445\\
Living with child only & 92.184 & 7.816\\
Living with spouse and child & 97.863 & 2.137\\
Living with others only & 93.046 & 6.954\\
\bottomrule
\end{tabular}
\end{table}

What you will immediately realize is that these numbers are way lower
than those when using coding scheme 1 (that is, the coding scheme of
\citet{lim_association_2017} and \citet{chan_loneliness_2015}). These
numbers are more consistent with the figures shown in
\citet{ge_social_2017}\footnote{Note that the sample in
  \citet{ge_social_2017} is of all adults aged 21 and older, not just
  older adults, so the lower number is expected. Note also that in the
  paper, the authors show column percentages instead of row percentages.
  Since we are comparing across living arrangementsm however, row
  percentages are more appropriate.}. Further, the difference in the
proportion of those living alone and those living with their child (but
without their spouse) is similar in absolute terms, but much smaller in
relative terms (see \ref{tab:table-lonely3}). While the proportion of
those lonely among those who live alone is 2 times that of those living
with only their children according to coding scheme 1, this ratio
reduces to 1.2 when using coding scheme 2. Overall, based on these
results, it seems that the loneliness situation is much less dire than
before.

\begin{table}[t]

\caption{\label{tab:table-lonely3}Comparison of absolute and relative differences}
\centering
\begin{tabular}{lrr}
\toprule
  & Coding scheme 1 & Coding scheme 2\\
\midrule
(1) Living alone & 15.6 & 59.2\\
(2) Living with child only & 7.8 & 50.1\\
Difference [(1) - (2)] & 7.8 & 9.1\\
Ratio [(1)/(2)] & 2.0 & 1.2\\
\bottomrule
\end{tabular}
\end{table}

\section{Conclusion}\label{conclusion-1}

So which way of conceptualizing loneliness is ``right''? As I mentioned
at the start of this case study, that is not the goal here. The process
of figuring out a useful cut-off point is a long and tedious one that
requires researchers to engage with each other\footnote{And Singapore
  social science can use more of this.}. Rather, the goal here has been
to highlight that decisions like cut-off points may seem small, but are
critical. In this case study, these cut-off points essentially define
who is considered lonely. \textbf{If these cut-off points are vastly
different, we are probably not even talking about the same people}. This
means that when comparing or drawing conclusions from the findings of
different studies, it is crucial that we understand how they were
arrived at.

\chapter{Case study 3}\label{case3}

\begin{verbatim}
Contributor: 
Date: 
\end{verbatim}

Another case study goes here. Do you wish to contribute? Send me an
email at
\href{mailto:shannon.ang@ntu.edu.sg}{\nolinkurl{shannon.ang@ntu.edu.sg}}

This is an example of in-line code annotation and output.

\begin{Shaded}
\begin{Highlighting}[]
\KeywordTok{par}\NormalTok{(}\DataTypeTok{mar =} \KeywordTok{c}\NormalTok{(}\DecValTok{4}\NormalTok{, }\DecValTok{4}\NormalTok{, .}\DecValTok{1}\NormalTok{, .}\DecValTok{1}\NormalTok{))}
\KeywordTok{plot}\NormalTok{(pressure, }\DataTypeTok{type =} \StringTok{'b'}\NormalTok{, }\DataTypeTok{pch =} \DecValTok{19}\NormalTok{)}
\end{Highlighting}
\end{Shaded}

\begin{figure}

{\centering \includegraphics[width=0.8\linewidth]{bookdown_files/figure-latex/nice-fig-1} 

}

\caption{Here is a nice figure!}\label{fig:nice-fig}
\end{figure}

Figures can be referenced, e.g., see Figure \ref{fig:nice-fig}.
Similarly, you can reference tables generated from
\texttt{knitr::kable()}, e.g., see Table \ref{tab:nice-tab}.

\begin{Shaded}
\begin{Highlighting}[]
\NormalTok{knitr}\OperatorTok{::}\KeywordTok{kable}\NormalTok{(}
  \KeywordTok{head}\NormalTok{(iris, }\DecValTok{5}\NormalTok{), }\DataTypeTok{caption =} \StringTok{'Here is a nice table!'}\NormalTok{,}
  \DataTypeTok{booktabs =} \OtherTok{TRUE}
\NormalTok{)}
\end{Highlighting}
\end{Shaded}

\begin{table}[t]

\caption{\label{tab:nice-tab}Here is a nice table!}
\centering
\begin{tabular}{rrrrl}
\toprule
Sepal.Length & Sepal.Width & Petal.Length & Petal.Width & Species\\
\midrule
5.1 & 3.5 & 1.4 & 0.2 & setosa\\
4.9 & 3.0 & 1.4 & 0.2 & setosa\\
4.7 & 3.2 & 1.3 & 0.2 & setosa\\
4.6 & 3.1 & 1.5 & 0.2 & setosa\\
5.0 & 3.6 & 1.4 & 0.2 & setosa\\
\bottomrule
\end{tabular}
\end{table}

\bibliography{book.bib,packages.bib,rest.bib}


\end{document}
